% ************************** Thesis Abstract *****************************
% Use `abstract' as an option in the document class to print only the titlepage and the abstract.
\begin{abstract}

%Tujuan utama Tim SAR adalah menemukan korban di lokasi pasca bencana alam secepat mungkin. Penelitian tentang robot yang membantu Tim SAR menemukan korban berkembang pesat. Sayangnya, penelitian sebelumnya hanya menguji coba proses deteksi korban pada latar belakang yang sederhana seperti di dalam ruangan atau di tanah lapang. Padahal deteksi korban bencana alam memiliki tingkat kesulitan yang kompleks, dimana citra kondisi manusia menyerupai citra lingkungan pasca bencana. Pada proyek akhir ini akan dilakukan penelitian tentang pendeteksian korban bencana alam dengan data citra yang diambil dari udara. Untuk meningkatkan akurasi deteksi, Deep Learning akan diterapkan bersama dengan framework Caffe pada pengolahan citra menggunakan arsitektur Convolutional Neural Networks (CNN). Deep Learning mampu melakukan feature extraction pada proses learning sehingga peneliti tidak perlu mendefinisikan feature yang akan digunakan. Penelitian ini juga akan membuat dataset khusus korban bencana alam yang merepresentasikan keadaan yang sebenarnya. Pada keluaran diharapkan mampu menemukan korban-korban dan menentukan area paling banyak korban. Dengan adanya pencarian udara yang secara otomatis menemukan lokasi korban, maka Tim SAR dapat langsung menuju lokasi tanpa resiko keselamatan tim dan inefisiensi waktu.

% Penelitian tentang robot kebencanaan berkembang pesat. Penelitian sebelumnya hanya menguji coba proses deteksi korban pada latar belakang yang sederhana seperti di dalam ruangan atau di tanah lapang. Deteksi korban bencana alam memiliki tingkat kesulitan yang kompleks, dimana citra kondisi manusia menyerupai citra lingkungan pasca bencana. Pada proyek akhir ini akan dilakukan penelitian tentang pendeteksian korban bencana alam dengan data citra yang diambil dari udara. Untuk meningkatkan akurasi deteksi, Deep Learning akan diterapkan pada pengolahan citra menggunakan arsitektur Convolutional Neural Networks (CNN). Deep Learning mampu melakukan \textit{feature extraction} pada proses \textit{learning} sehingga \textit{feature} tidak mengandalkan peran expert.

% Pada proyek akhir ini telah dilakukan pengujian kemampuan pengambilan dan pengiriman gambar serta melakukan \textit{learning} dan \textit{testing} pada berbagai gambar. Pengujian menggunakan empat resolusi gambar dan dilihat ukuran data, waktu dan akurasi. Pengambilan gambar resolusi 800x600 merupakan data paling optimal karena memiliki ukuran data yang relatif kecil, waktu deteksi yang singkat tetapi dengan akurasi deteksi yang relatif tinggi. Selanjutnya akan dibuat dataset untuk \textit{re-training}, pengujian dengan pose korban yang berbeda dan pengujian pada satu peta utuh yang dibuat saat penjelajahan udara.

Indonesia memiliki potensi terjadinya bencana alam yang besar, salah satunya gempa bumi yang hingga saat ini belum dapat diprediksi. Penelitian ini menerapkan teknologi kecerdasan buatan pada unmanned aerial vehicle (UAV) untuk meningkatkan kecepatan proses pencarian korban. Deep Learning Convolutional Neural Networks (CNN) dipilih sebagai metode kecerdasan buatan untuk menklasifikasikan korban pada seluruh data gambar yang diambil oleh UAV lalu dibuatkan sebuah peta lokasi secara real-time dan akurat menggunakan GPS. Proses pengambilan gambar, pengiriman gambar dan klasifikasi korban berturut-turut membutuhkan waktu Xs, Ys, dan Zs. Data yang diambil pada satu waktu dapat mencapai 25 fps dengan kecepatan terbang UAV rata-rata V m/s. Hasil penelitian menunjukan perhitungan kebutuhan waktu pencarian yang singkat bergantung pada luas daerah terdampak.


{\noindent \textbf{Kata Kunci:} \textit{Deteksi Korban Bencana Alam, Deep Learning, Convolutional Neural Networks (CNN), Pengambilan Gambar Udara.}}

\end{abstract}
