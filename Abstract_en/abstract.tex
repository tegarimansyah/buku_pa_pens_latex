% ************************** Thesis Abstract *****************************
% Use `abstract' as an option in the document class to print only the titlepage and the abstract.
\begin{abstract_en}
% Research on disaster robot that helps to find victims growing rapidly. Unfortunately, previous studies have only tested a toll on the detection process simple background such as indoors or in the field. Though detection of victims of natural disasters has a difficulty level of the complex, where the image of the human condition resembles the image of post-disaster environment. At the end of this project will do research on the detection of victims of natural disasters with image data taken from the air. To improve detection accuracy, Deep Learning will be implemented together with the framework Caffe on image processing architecture using Convolutional Neural Networks (CNN). Deep Learning is able to perform feature extraction in the learning process so that researchers do not need to define a feature that will be used. 

% The results for progress of this final project are we were able to know the ability of taking and sending pictures and try to run learning and testing algorithm on different images. Tests performed by four images resolution to look the data size, timing and accuracy. Images with a resolution of 800x600 is considered the most optimal because it has a relatively small data size, the detection time is short but with a relatively high detection accuracy. Furthermore, the dataset will be made to do a re-training, testing with different victim poses and testing on a single large map made while traversing the air.

Indonesia memiliki potensi terjadinya bencana alam yang besar, salah satunya gempa bumi yang hingga saat ini belum dapat diprediksi. Penelitian ini menerapkan teknologi kecerdasan buatan pada unmanned aerial vehicle (UAV) untuk meningkatkan kecepatan proses pencarian korban. Deep Learning Convolutional Neural Networks (CNN) dipilih sebagai metode kecerdasan buatan untuk menklasifikasikan korban pada seluruh data gambar yang diambil oleh UAV lalu dibuatkan sebuah peta lokasi secara real-time dan akurat menggunakan GPS. Proses pengambilan gambar, pengiriman gambar dan klasifikasi korban berturut-turut membutuhkan waktu Xs, Ys, dan Zs. Data yang diambil pada satu waktu dapat mencapai 25 fps dengan kecepatan terbang UAV rata-rata V m/s. Hasil penelitian menunjukan perhitungan kebutuhan waktu pencarian yang singkat bergantung pada luas daerah terdampak.


%Keywords: \textbf{Disaster Victim Detection, Deep Learning, Caffe, Convolutional Neural Networks (CNN), Disaster Victim Dataset, Aerial Photography.}
{\noindent \textbf{Keywords:} \textit{Disaster Victim Detection, Deep Learning, Convolutional Neural Networks (CNN), Aerial Photography.}}
\end{abstract_en}
